\documentclass{article}
\usepackage[utf8]{inputenc}
\usepackage{graphicx}
\usepackage{hyperref}
\usepackage{fancyhdr}
\usepackage{amsfonts}
\usepackage{amsmath}
\usepackage{amsthm}
\usepackage{amssymb}
\usepackage{enumitem}
\usepackage{listings}
\usepackage{tikz}
\usepackage[
    letterpaper,
    margin=0.4in,
    tmargin=0.7in,
    bmargin=0.7in,
    headsep=12pt,
    footskip=12pt
]{geometry}

\theoremstyle{definition}
\newtheorem*{ques}{Question}

\renewcommand{\Re}{\operatorname{Re}}
\newcommand{\diam}{\operatorname{diam}}
\newcommand{\RR}{\mathbb{R}}
\newcommand{\trace}{\operatorname{trace}}
\newcommand{\kleene}{^\ast}
\newcommand{\SUBSTRING}{\text{SUBSTRING}}
\newcommand{\REP}{\text{REP}}

% <Laplace Transform>
\usepackage{mathrsfs}
\newsavebox\foobox
\newlength{\foodim}
\newcommand{\slantbox}[2][0]{\mbox{%
        \sbox{\foobox}{#2}%
        \foodim=#1\wd\foobox
        \hskip \wd\foobox
        \hskip -0.5\foodim
        \pdfsave
        \pdfsetmatrix{1 0 #1 1}%
        \llap{\usebox{\foobox}}%
        \pdfrestore
        \hskip 0.5\foodim
}}
\def\Laplace{\slantbox[-.45]{$\mathscr{L}$}}
% </Laplace Transform>

\pagestyle{fancy}
\fancyhf{}
\rhead{CSE 105}
\lhead{
    Devanshi Jain, Matthew Stringer 
}
\chead{HW 4}
\cfoot{\thepage}

\title{CSE 105 HW 4}
\author{
    Devanshi Jain, Matthew Stringer 
}
% \date{}

\renewcommand{\ques}[1]{\section*{Question #1.}}

\begin{document}
    \maketitle
    \ques{1}
    \paragraph{(a)}
        \textbf{Answer: } 
    \paragraph*{(b)}
        \textbf{Answer: }  
    \paragraph*{(c)}
        \textbf{Answer: } 
    \paragraph*{(d)}
        \textbf{Answer: }
    
    \ques{2}
    \paragraph*{(a)}
        \textbf{Answer: } We can define the following context-free 
        grammar for $\REP({0^n1^n \mid n \ge 0})$:
        \begin{align*}
        S &\to \epsilon \mid 2AS2\\
        A &\to 0A1 \mid \epsilon
        \end{align*}
        Intuitively, $S$ generates all strings in 
        $\REP({0^n1^n \mid n \ge 0})$ by allowing the production of 
        the empty string or strings that begin and end with $2$ and 
        have a sequence of $0$s and $1$s in between. The $A$ 
        nonterminal generates any valid sequence of $0$s and $1$s.
        Examples:
        \begin{itemize}
        \item $w = \epsilon$ is in $\REP(\{0^n1^n \mid n \ge 0\})$
        with the derivation:
        \begin{align*}
        S &\Rightarrow \epsilon
        \end{align*}
                \item $w = 200112$ is in $\REP(\{0^n1^n \mid n \ge 0\})$
                with the derivation:
        \begin{align*}
        S &\Rightarrow 2AS2 \Rightarrow 20A1S2 \Rightarrow 200A11S2 
        \Rightarrow 200\epsilon11\epsilon22 \Rightarrow 2001122
        \end{align*}

        \item $w = 20102$ is not in $\REP(\{0^n1^n \mid n \ge 0\})$
         because for 2v2 to be in $\SUBSTRING(w)$, $v$ must be in 
         $\{0^n1^n \mid n \ge 0\}$, which it clearly isn't. 
         Attempting to derive $w$ results in the following:
        \begin{align*}
            S &\Rightarrow 2AS2 \Rightarrow 20A1S2 \Rightarrow 
            20\epsilon1S2 \Rightarrow 201S2
        \end{align*}
        Note: At no productions of $S$ can we derive a $0$, $qed$.
    \end{itemize}

    \paragraph*{(b)}
        \textbf{Answer: } 

    \ques{3}
    \paragraph*{(a)} 
        \textbf{Answer: } 
    \paragraph*{(b)}
        \textbf{Answer: }
    \paragraph*{(c)}
        \textbf{Answer: }
\end{document}